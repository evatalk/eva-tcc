% Resumo
\begin{center}
	{\Large{\textbf{EVA: Desenvolvimento de um \textit{Chatbots} para Escola Virtual do Governo}}}
\end{center}

\vspace{1cm}

\begin{flushright}
	Autor: Mauro de Carvalho Gonçalves\\
	Orientador(a): Leonardo Ataide Minora
\end{flushright}

\vspace{1cm}

\begin{center}
	\Large{\textsc{\textbf{Resumo}}}
\end{center}

\noindent A Escola Virtual do Governo (EV.G) é uma iniciativa que consiste em um portal único para oferta de capacitação a distância voltado a servidores públicos e cidadãos de todo o país.
A EV.G oferece cursos a distância de diferentes instituições e com diferentes temáticas ligadas à administração pública e a cidadania, viabilizando o desafio de contribuir para a formação e o desenvolvimento de milhares de servidores públicos bem como dos cidadãos.
Contudo, o EV.G ainda não dispõe de uma central de atendimento aos alunos propriamente dita, e em ambientes de oferta massiva de cursos, o atendimento ao usuário pode ser considerado um diferencial competitivo.
Com o avanço tecnológico, soluções inovadoras e de baixo custo estão surgindo a todo momento. No cenário do atendimento on-line, o uso de \textit{chatbots} é uma opção de baixo custo e alto desempenho. 
Neste contexto de atendimento massivo, foi desenvolvido um \textit{chatbot} de conversação textual para a plataforma de mensagens instantâneas chamada Telegram, denominado de EVA (EV.G Virtual Assistant). O objetivo geral de EVA é ter a capacidade de conversar via texto com alunos da EV.G no atendimento administrativo de uma secretaria acadêmica.


\noindent\textit{Palavras-chave}: \textit{Chatbot}, Atendimento ao cliente, Atendimento on-line.