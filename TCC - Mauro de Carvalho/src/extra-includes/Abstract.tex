% Resumo em l�ngua estrangeira (em ingl�s Abstract, em espanhol Resumen, em franc�s R�sum�)
\begin{center}
	{\Large{\textbf{EVA: Development of a Chatbot for Escola Virtual de Governo}}}
\end{center}

\vspace{1cm}

\begin{flushright}
	Author: Mauro de Carvalho Gonçalves\\
	Supervisor: Leonardo Ataide Minora
\end{flushright}

\vspace{1cm}

\begin{center}
	\Large{\textsc{\textbf{Abstract}}}
\end{center}

\noindent The Escola Virtual do Governo (EV.G) is an initiative that consists of a single portal to offer distance training aimed at public servants and citizens from all over the country.
Launched in 2018, EV.G offers distance courses from different institutions and on different topics related to public administration and citizenship, making it possible to contribute to the training and development of thousands of public servants as well as citizens.
Although it is an environment of mass offer of courses (MOOC), in which user service can be considered a competitive differential, EV.G does not yet have a service center for the students as such.
In the online service scenario, the use of \ textit {chatbots} is a low-cost, high-performance option.
In this context, it was developed by the author together with the EV.G, a textual conversation chatboot compatible with the instant messaging platform called Telegram, called EVA (EV.G Virtual Assistant). The general objective of EVA is to automate the interaction through text in the first level administrative attendance to students within the scope of an academic secretariat.

\noindent\textit{Keywords}: \textit{Chatbot}, Customer Service, Online Customer Service, MOOC.