% Resumo em l�ngua estrangeira (em ingl�s Abstract, em espanhol Resumen, em franc�s R�sum�)
\begin{center}
	{\Large{\textbf{EVA: Development of a Chatbot for Escola Virtual do Governo}}}
\end{center}

\vspace{1cm}

\begin{flushright}
	Author: Mauro de Carvalho Gonçalves\\
	Supervisor: Leonardo Ataide Minora
\end{flushright}

\vspace{1cm}

\begin{center}
	\Large{\textsc{\textbf{Abstract}}}
\end{center}

\noindent The Escola Virtual do Governo (EV.G) is an initiative that consists of a single portal to offer distance training aimed at public servants and citizens from all over the country.
EV.G offers distance courses from different institutions and with different themes related to public administration and citizenship, making it possible to contribute to the training and development of thousands of civil servants as well as citizens.
However, the EV.G still does not have a student service center per se, and in environments where there are massive courses, user service can be considered a competitive differential.
With technological advancement, innovative and low-cost solutions are emerging at all times. In the online service scenario, the use of \textit{chatbots} is a low-cost, high-performance option.
In this context of mass service, a textual conversation chatbot was developed for the instant messaging platform called Telegram called the EVA (EV.G Virtual Assistant). The general objective of EVA is to have the ability to converse via text with EV.G students in the administrative service of an academic secretariat.

\noindent\textit{Keywords}: \textit{Chatbot}, Customer Service, Online Customer Service.