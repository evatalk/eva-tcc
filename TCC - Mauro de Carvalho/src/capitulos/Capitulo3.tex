\chapter{Desenvolvimento de EVA}

\section{Apresentação do projeto}


A EVA (EV.G Virtual Assistant) é um \textit{chatbot} de domínio amplo que será capaz de interagir com os alunos vinculados ao EV.G na medida de suas necessidades.
Ela irá atender tais solicitações via mecanismo de conversação textual na plataforma de mensagens instantâneas chamada Telegram, no atendimento administrativo de uma secretaria acadêmica.

Por meio de EVA, os alunos vinculados ao EV.G poderão usufruir de várias funcionalidades.
Eles irão gerenciar seus cursos visualizando o andamento das inscrições, terão acesso ao catálogo unificado, calendário de turmas, histórico escolar e também serão auxiliados no processo de emissão de certificado. Tudo por meio de um acesso único e simplificado.

\subsection{Especificação dos requisitos de EVA}

É interessante deixar claro que todo o processo de levantamento e validação dos requisitos foi realizado e entregue por parte da EV.G, assim, não haverá nenhuma seção que aborde como se deu tal levantamento. Ao invés disso, os requisitos de EVA serão apenas especificados nas subseções a seguir.

\subsubsection{Requisitos funcionais}

\subsubsection{Requisitos não funcionais}

\section{Arquitetura de EVA}

\subsection{Arquitetura da API}

\subsection{Arquitetura do \textit{chatbot}}

\section{Modelagem dos diálogos de EVA}

\subsection{Desenvolvimento dos diálogos}