\chapter{Desenvolvimento de EVA}

\section{Apresentação do projeto}

A EVA (EV.G Virtual Assistant) é um \textit{chatbot} de domínio amplo que será capaz de interagir com os alunos vinculados ao EV.G na medida de suas necessidades. Ela irá atender tais solicitações via mecanismo de conversação textual na plataforma de mensagens instantâneas chamada Telegram, no atendimento administrativo de uma secretaria acadêmica.

Por meio de EVA, os alunos vinculados ao EV.G poderão usufruir de várias funcionalidades. Eles irão gerenciar seus cursos visualizando o andamento das inscrições, terão acesso ao catálogo unificado, calendário de turmas, histórico escolar e também serão auxiliados no processo de emissão de certificado. Tudo por meio de um acesso único e simplificado.

\subsection{Especificação dos requisitos}

Como sistemas computacionais são construídos para terem utilidade no mundo real, modelar o domínio de aplicação é uma atividade extremamente importante. A partir dessa atividade, pode se compreender a necessidade e a importância do sistema a ser construído e também definir os requisitos que tornam o sistema útil.

De acordo com \cite{} requisitos são objetivos ou restrições estabelecidas por clientes e usuários do sistema que definem as diversas propriedades do sistema. Os requisitos  de software são, obviamente, aqueles dentre os requisitos de sistema que dizem respeito a propriedades do software.

Um conjunto de requisitos pode ser definido como uma condição ou capacidade necessária que o software deve possuir (1) para que o usuário possa resolver um problema ou atingir um objetivo ou (2) para atender as necessidades ou restrições da organização ou dos outros componentes do sistema.

\subsubsection{Requisitos funcionais}

Os requisitos funcionais são a descrição das diversas funções que clientes e usuários querem ou precisam que o software ofereça. Eles definem a funcionalidade desejada do software. O termo função é usado no sentido genérico de operação que pode ser realizada pelo sistema, seja através comandos dos usuários ou seja pela ocorrência de eventos internos ou externos ao sistema.


\subsubsection{Requisitos não-funcionais}

Requisitos não-funcionais são as qualidades globais de um software, como manutenibilidade, usabilidade, desempenho, custos e várias outras. Normalmente estes requisitos são descritos de maneira informal, de maneira controversa (por exemplo, o gerente quer segurança mas os usuários querem facilidade de uso) e são difíceis de validar.

\section{Arquitetura de EVA}

\subsection{Arquitetura da API}

\subsection{Arquitetura do \textit{chatbot}}

\section{Modelagem dos diálogos de EVA}

\subsection{Boas práticas}

\subsection{Desenvolvimento dos diálogos}