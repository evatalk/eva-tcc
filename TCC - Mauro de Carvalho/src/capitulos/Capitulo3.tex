\chapter{Desenvolvimento de EVA}

\section{Apresentação do projeto}


A EVA (EV.G Virtual Assistant) é um \textit{chatbot} de domínio amplo que será capaz de interagir com os alunos vinculados ao EV.G na medida de suas necessidades.
Ela irá atender tais solicitações via mecanismo de conversação textual na plataforma de mensagens instantâneas chamada Telegram, no atendimento administrativo de uma secretaria acadêmica.

Por meio de EVA, os alunos vinculados ao EV.G poderão usufruir de várias funcionalidades.
Eles irão gerenciar seus cursos visualizando o andamento das inscrições, terão acesso ao catálogo unificado, calendário de turmas, histórico escolar e também serão auxiliados no processo de emissão de certificado. Tudo por meio de um acesso único e simplificado.

\section{Especificação dos requisitos de EVA}\label{especificacao-requisitos-eva}

Como dito \ref{cap:02:sec:05:projeto}, modelar o domínio de aplicação de um \textit{chatbot} é uma atividade de suma importância, pois, a partir desta atividade, pode se compreender a necessidade do sistema a ser construído, e também, definir os requisitos que o tornam útil.
É interessante deixar claro, que todo o processo de levantamento e validação dos requisitos foi realizado e entregue por parte da EV.G. Assim, não haverá nenhuma seção que aborde como se deu tal levantamento. Ao invés disso, os requisitos de EVA serão apenas especificados nas subseções a seguir.

\subsection{Requisitos funcionais}

Na seção \ref{cap:02:sec:05:projeto:classificacao-requisitos}, foi explicado o que são os requisitos funcionais de um sistema. Abaixo, estão listados os requisitos funcionais elencados para o sistema de EVA.

\begin{table}[htb]
\textsf{\caption{Especificação de requisitos funcionais de EVA}}\label{table:tabela1}
\center
\footnotesize
\begin{tabular}{|p{1.2cm}|p{3.5cm}|p{7.5cm}|}
  \hline
   \textbf{\#RF} & \textbf{Nome}  & \textbf{Descrição}  \\
    \hline
    RF01 & Conversação & Permitir que o usuário interaja com o \textit{chatbot} na língua portuguesa por meio de mensagens textuais em um \textit{messenger}. \\
    \hline
    RF02 & Compreensão & O \textit{chatbot} deve ser capaz de compreender o que o usuário solicita, através das mensagens textuais enviadas por ele, e tomar as devidas decisões para atende-lo. \\
    \hline
    RF03 & Autenticação & O \textit{chatbot} deve possuir uma forma de identificar o usuário que está solicitando acesso à aplicação. \\
   \hline
     RF04 & Visualizar o histórico escolar & O \textit{chatbot} deve informar ao usuário o seu histórico escolar completo caso seja solicitado por ele. \\
   \hline
    RF05 & Auxiliar na emissão de certificados & O \textit{chatbot} deve auxiliar o usuário no processo de emissão de certificados dos cursos concluídos por ele, caso seja solicitado por ele. \\
   \hline
    RF06 & Visualizar as inscrições de cursos & O \textit{chatbot} deve exibir ao usuário quais inscrições de cursos que estão abertas, caso seja solicitado por ele. \\
   \hline
   
\end{tabular}
\end{table}

\subsection{Requisitos não funcionais}

Na seção \ref{cap:02:sec:05:projeto:classificacao-requisitos}, foi explicado o que são os requisitos não funcionais de um sistema. Abaixo, estão listados os requisitos funcionais elencados para o sistema de EVA.

\begin{table}[htb]
\textsf{\caption{Especificação de requisitos não funcionais de EVA}}
\center
\footnotesize
\begin{tabular}{|p{1.2cm}|p{3.5cm}|p{7.5cm}|}
  \hline
   \textbf{\#RNF} & \textbf{Nome}  & \textbf{Descrição}  \\
   \hline
    RNF01 & Disponibilidade & O sistema deve estar disponível continuamente (24 horas / 7 dias por semana). \\
   \hline
    RNF02 & Confidencialidade & O sistema deve garantir a visualização dos dados apenas pelo usuário associado. \\
   \hline
    RNF03 & Integridade & O sistema deve garantir que caso algum usuário falhe ao tentar se autenticar por três vezes consecutivas, bloqueie as tentativas de acesso dele durante 24 horas. \\
   \hline
   RNF04 & Portabilidade & O sistema deverá atender prioritariamente na plataforma de mensagens instantâneas Telegram. \\
   \hline
    RNF05 & Interoperabilidade & O sistema deverá se comunicar com algum serviço de Processamento de Linguagem Natural. \\
   \hline
    RNF06 & Implementação & O sistema deverá ser implementado utilizando a linguagem de programação Python. \\
   \hline
   
\end{tabular}
\end{table}\label{tabela:3}

\section{Especificação dos casos de uso}

Na seção \ref{texto:especificando-com-casos-de-uso}, foi descrito o que são casos de uso e como eles normalmente são utilizados durante o projeto de um sistema. Elaborar os casos de uso permite definir quais funções da aplicação que o sistema deverá oferecer para o usuário. Para a sua especificação, serão utilizados alguns dos requisitos funcionais e não funcionais, identificados na seção \ref{especificacao-requisitos-eva} para descrever a funcionalidade do sistema.

\subsection{Identificação dos atores}

\subsection{Identificação dos casos de uso}

\subsection{Diagrama de casos de uso}

\subsection{Detalhamento dos casos de uso}

\section{Modelagem dos diálogos de EVA}

\subsection{Cenários}

\section{Arquitetura de EVA}

\subsection{Arquitetura da API}

\subsection{Arquitetura do \textit{chatbot}}