\chapter{Desenvolvimento de EVA}

\section{Apresentação do projeto}


A EVA (EV.G Virtual Assistant) é um \textit{chatbot} de domínio amplo que será capaz de interagir com os alunos vinculados ao EV.G na medida de suas necessidades.
Ela irá atender tais solicitações via mecanismo de conversação textual na plataforma de mensagens instantâneas chamada Telegram, no atendimento administrativo de uma secretaria acadêmica.

Por meio de EVA, os alunos vinculados ao EV.G poderão usufruir de várias funcionalidades.
Eles irão gerenciar seus cursos visualizando o andamento das inscrições, terão acesso ao catálogo unificado, calendário de turmas, histórico escolar e também serão auxiliados no processo de emissão de certificado. Tudo por meio de um acesso único e simplificado.

\section{Detalhamento dos requisitos de EVA}\label{especificacao-requisitos-eva}

Como dito \ref{cap:02:sec:05:projeto}, modelar o domínio de aplicação de um \textit{chatbot} é uma atividade de importante, pois, a partir desta atividade, pode se compreender a necessidade do sistema a ser construído, e também, definir os requisitos que o tornam útil.
É interessante deixar claro, que todo o processo de levantamento e validação dos requisitos foi realizado e entregue por parte da EV.G. Assim, não haverá nenhuma seção que aborde como se deu tal levantamento. Ao invés disso, os requisitos de EVA serão apenas detalhados nas subseções a seguir.

\subsection{Requisitos funcionais}

Na seção \ref{cap:02:sec:05:projeto:classificacao-requisitos}, foi explicado o que são os requisitos funcionais de um sistema. Abaixo, na Tabela \ref{tabela:tabela1}, estão listados os requisitos funcionais elencados para o sistema de EVA.

\begin{table}[htb]
\caption{Detalhamento de requisitos funcionais de EVA}
% \textsf{\caption{Especificação de requisitos funcionais de EVA}}
\label{tabela:tabela1}
% \center
% \footnotesize
\centering
\medskip
\begin{tabular}{|p{1.2cm}|p{3.5cm}|p{7.5cm}|}
  \hline
   \textbf{\#RF} & \textbf{Nome}  & \textbf{Descrição}  \\
    \hline
    RF01 & Conversação & Permitir que o usuário interaja com o \textit{chatbot} na língua portuguesa por meio de mensagens textuais em um \textit{messenger}. \\
    \hline
    RF02 & Compreensão & O \textit{chatbot} deve ser capaz de compreender o que o usuário solicita, através das mensagens textuais enviadas por ele, e tomar as devidas decisões para atende-lo. \\
    \hline
    RF03 & Autenticação & O \textit{chatbot} deve possuir uma forma de identificar o usuário que está solicitando acesso à aplicação. \\
   \hline
     RF04 & Visualizar o histórico escolar & O \textit{chatbot} deve informar ao usuário o seu histórico escolar completo caso seja solicitado por ele. \\
   \hline
    RF05 & Auxiliar na emissão de certificados & O \textit{chatbot} deve auxiliar o usuário no processo de emissão de certificados dos cursos concluídos por ele, caso seja solicitado por ele. \\
   \hline
    RF06 & Visualizar as inscrições de cursos & O \textit{chatbot} deve exibir ao usuário quais inscrições de cursos que estão abertas, caso seja solicitado por ele. \\
   \hline
\end{tabular}
\end{table}

\subsection{Requisitos não funcionais}

Na seção \ref{cap:02:sec:05:projeto:classificacao-requisitos}, foi explicado o que são os requisitos não funcionais de um sistema. Abaixo, na Tabela \ref{tabela:tabela2}, estão listados os requisitos não funcionais elencados para o sistema de EVA.

\begin{table}[htb!]
\caption{Detalhamento de requisitos não funcionais de EVA}
\label{tabela:tabela2}
\center
\footnotesize
\begin{tabular}{|p{1.2cm}|p{3.5cm}|p{7.5cm}|}
  \hline
   \textbf{\#RNF} & \textbf{Nome}  & \textbf{Descrição}  \\
   \hline
    RNF01 & Disponibilidade & O sistema deve estar disponível continuamente (24 horas / 7 dias por semana). \\
   \hline
    RNF02 & Confidencialidade & O sistema deve garantir a visualização dos dados apenas pelo usuário associado. \\
   \hline
    RNF03 & Integridade & O sistema deve garantir que caso algum usuário falhe ao tentar se autenticar por três vezes consecutivas, bloqueie as tentativas de acesso dele durante 24 horas. \\
   \hline
   RNF04 & Portabilidade & O sistema deverá atender prioritariamente na plataforma de mensagens instantâneas Telegram. \\
   \hline
    RNF05 & Interoperabilidade & O sistema deverá se comunicar com algum serviço de Processamento de Linguagem Natural. \\
   \hline
    RNF06 & Implementação & O sistema deverá ser implementado utilizando a linguagem de programação Python. \\
   \hline
   
\end{tabular}
\end{table}\label{tabela:3}

\section{Casos de uso de EVA}

Na seção \ref{texto:especificando-com-casos-de-uso}, foi descrito o que são casos de uso e como eles normalmente são utilizados durante o projeto de um sistema. Elaborar os casos de uso permite definir quais funções de aplicação que o sistema deverá oferecer ao usuário~\cite{ReqJair}. Para a sua especificação, serão utilizados também alguns dos requisitos funcionais e não funcionais, que foram detalhados na seção \ref{especificacao-requisitos-eva}, para que se possa descrever as funcionalidades do sistema com ainda mais propriedade. Nas subseções a seguir, será realizado todo o processo de especificação e detalhamento dos casos de uso de EVA.

\subsection{Especificação dos atores}

Um ator é algo com comportamento, tal como uma pessoa (identificada por seu papel), um sistema ou uma organização~\cite{CraigLarman}. No que se diz respeito a EVA, foram identificados como atores o Visitante e o Aluno. Ambos poderão interagir com EVA, por meio de mensagens textuais, porém com as devidas restrições. É válido ressaltar, que a EVA não entra como ator, já que será o próprio sistema na qual está sendo modelando. Na Tabela \ref{tabela:tabela3} os atores são especificados.

\begin{table}[htb!]
\caption{Especificação dos atores do sistema de EVA}
\label{tabela:tabela3}
\center
\footnotesize
\begin{tabular}{|p{2cm}|p{3cm}|p{7.5cm}|}
  \hline
   \textbf{\#ATOR} & \textbf{Nome}  & \textbf{Descrição}  \\
   \hline
    ATOR01 & Visitante & Qualquer pessoa que interaja com EVA sem estar devidamente autenticado e identificado no sistema. \\
   \hline
    ATOR02 & Aluno & Qualquer pessoa que interaja com EVA e esteja devidamente autenticado e identificado no sistema. \\
   \hline
\end{tabular}
\end{table}


\subsection{Especificação dos casos de uso}

Como dito no começo desta seção, casos de uso permitem definir as funções de aplicação que o sistema deverá oferecer para o usuário. A partir dos requisitos funcionais identificados para EVA, foram extraídos os casos de uso especificados na Tabela \ref{tabela:tabela4}.

\begin{table}[htb!]
\caption{Especificação dos casos de uso de EVA}
\label{tabela:tabela4}
\center
\footnotesize
\begin{tabular}{|p{2cm}|p{3cm}|p{7.5cm}|}
  \hline
   \textbf{\#UC} & \textbf{Nome}  & \textbf{Descrição}  \\
   \hline
    UC01 & Dialogar com EVA & Os usuários do sistema poderão dialogar com EVA por meio de mensagens textuais, na língua portuguesa, através de um \textit{messenger}. A partir dessa funcionalidade, o usuário irá acessar todas as demais.\\
   \hline
    UC02 & Efetuar login & Autenticação de um usuário, permitindo que ele tenha acesso às funcionalidades restritas de EVA. \\
   \hline
    UC03 & Visualizar histórico escolar completo & O usuário poderá solicitar a visualização do histórico de todos os cursos que ele esteve matriculado no âmbito da EV.G. \\
   \hline
    UC04 & Receber auxílio na emissão de certificados & O usuário poderá solicitar auxilio para a emissão dos certificados dos cursos no qual ele finalizou no âmbito da EV.G. \\
   \hline
    UC05 & Visualizar inscrições de cursos abertos & O usário poderá solicitar a visualização de todos os cursos na qual possui matricula ativa no âmbito da EV.G. \\
   \hline
    UC06 & Efetuar logout & O usuário deixará de estar autenticado no sistema. Com isso, ele volta a possuir acesso limitado às funcionalidades de EVA.\\
   \hline
\end{tabular}
\end{table}\label{tabela:3}

\subsection{Diagrama de casos de uso}

Para descrever de forma visual e clara como se dará o vínculo entre os atores e os casos de uso identificados de EVA, foi criado um diagrama de casos de uso. Para a criação deste, foi utilizada a notação UML.

Na UML, os atores são representados como figuras ‘palito’. Cada caso de uso, que são as possíveis interações que poderão ser realizadas, é representada por uma elipse. As linhas fazem a ligação entre os atores e as interações. Na Figura \ref{cap:03:fig:diagrama}, está representado o diagrama de casos de uso de EVA.

\begin{figure}
\centering
\includegraphics[width=0.7\linewidth]{src/imagens/CasoDeUsoEva.png}
\caption{Fonte: Elaborado pelo autor (2018)}
\label{cap:03:fig:diagrama}
\end{figure}

\subsection{Detalhamento dos casos de uso}

\subsubsection{UC01 - Dialogar com EVA}
\textbf{Descrição:}

\begin{itemize}
    \item \textbf{Ator(es):}
    \item \textbf{Requisitos funcionais:}
    \item \textbf{Requisitos não funcionais:}
\end{itemize}

\textbf{Fluxo básico:}

\subsubsection{UC02 - Efetuar login}
\textbf{Descrição:}

\begin{itemize}
    \item \textbf{Ator(es):}
    \item \textbf{Pré-condições}
    \item \textbf{Pos-condições}
    \item \textbf{Requisitos funcionais:}
    \item \textbf{Requisitos não funcionais:}
\end{itemize}

\textbf{Fluxo básico:}
\textbf{Fluxo alternativo A:}
\textbf{Fluxo alternativo B:}

\subsubsection{UC03 - Visualizar histórico escolar completo}
\textbf{Descrição:}

\begin{itemize}
    \item \textbf{Ator(es):}
    \item \textbf{Pré-condições}
    \item \textbf{Requisitos funcionais:}
    \item \textbf{Requisitos não funcionais:}
\end{itemize}

\textbf{Fluxo básico:}
\textbf{Fluxo alternativo A:}

\subsubsection{UC04 - Receber auxílio na emissão de certificados}
\textbf{Descrição:}

\begin{itemize}
    \item \textbf{Ator(es):}
    \item \textbf{Pré-condições}
    \item \textbf{Requisitos funcionais:}
    \item \textbf{Requisitos não funcionais:}
\end{itemize}

\textbf{Fluxo básico:}
\textbf{Fluxo alternativo A:}

\subsubsection{UC05 - Visualizar inscrições de cursos abertos}
\textbf{Descrição:}

\begin{itemize}
    \item \textbf{Ator(es):}
    \item \textbf{Pré-condições}
    \item \textbf{Requisitos funcionais:}
    \item \textbf{Requisitos não funcionais:}
\end{itemize}

\textbf{Fluxo básico:}
\textbf{Fluxo alternativo A:}

\subsubsection{UC06 - Efetuar logout}
\textbf{Descrição:}

\begin{itemize}
    \item \textbf{Ator(es):}
    \item \textbf{Pré-condições}
    \item \textbf{Pos-condições}
    \item \textbf{Requisitos funcionais:}
    \item \textbf{Requisitos não funcionais:}
\end{itemize}

\textbf{Fluxo básico:}

\section{Modelagem dos diálogos de EVA}

\subsection{Cenários}

\section{Arquitetura de EVA}

\subsection{Arquitetura da API}

\subsection{Arquitetura do \textit{chatbot}}