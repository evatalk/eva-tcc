% Capítulo 2
\chapter{\textit{Chatbots}: robôs de conversação}\label{cap:02:referencial}


\section{Contexto inicial}\label{cap:02:sec:01:contexto}

Em uma tradução literal do inglês, o termo refere-se às palavras “conversa” (\textit{chat}) e “robô” (\textit{bot} é abreviação de \textit{robot}) \cite{SimplyChat}.
Numa definição mais técnica, os \textit{chatbots} são softwares de respostas automáticas, programados para executar tarefas pré-definidas \cite{GamaMedeiros}.
Geralmente, eles estão vinculados a alguma aplicação de mensagens instantâneas, podendo atender demandas via textos, áudios, imagens entre outros formatos.
Alguns autores como~\alusao{Juliano} dividem os \textit{chatbots} em dois tipos: os baseados em regras, ou diálogos estruturados, e os de domínio amplo. Nas subseções abaixo, estes serão detalhados.


\subsection{\textit{Chatbots} baseados em regras}\label{cap:02:sec:01:sub:01:bot-regras}

\textit{Chatbots} baseados em regras, são geralmente relacionados a um domínio de conhecimento restrito~\cite{Juliano}.
Isso significa, que eles precisam desempenhar alguma função específica em uma determinada área, como por exemplo, em atendimentos ao cliente, suporte técnico, vendas de produtos, marketing, agendamentos, reservas, entre outros.
Considerando que essas áreas possuem funções pré-definidas, o conjunto de regras e ações podem ser previstos e configurados para esses \textit{chatbots}, utilizando, por exemplo, fluxogramas.

Assim, o usuário conversando com o \textit{chatbot} passa por um conjunto de perguntas ou opções e, com base nas respostas ou entradas, percorre o caminho predefinido~\cite{Samer}, como mostrado na figura~\ref{cap:01:fig:fluxograma}.

\begin{figure}
\centering
\includegraphics[width=0.9\linewidth]{src/imagens/Chatbot.png}
\caption{Chatbot Flowchart. Fonte:~\cite{Samer}}
\label{cap:01:fig:fluxograma}
\end{figure}

Outra característica importante nesse tipo de \textit{chatbots}, é que eles não precisam necessariamente compreender a linguagem humana para executarem suas tarefas~\cite{Juliano}.

\subsection{\textit{Chatbots} de domínio amplo}\label{cap:02:sec:01:sub:02:bot-dominio}

Já os \textit{chatbots} de domínio amplo utilizam de recursos mais avançados de Inteligência Artificial (\abrv[Inteligência Artificial]{AI}), são capazes de compreender o que o usuário solicita e podem relacionar-se a qualquer área de domínio de conhecimento~\cite{Juliano}. 
Na maior parte do tempo, o atendimento realizado por esse tipo de \textit{chatbots}, procura simular a conversação em linguagem humana.
Em outras palavras, um dos objetivos principais é responder perguntas de forma que os usuários tenham a impressão de estarem conversando com outra pessoa e não com um programa de computador.
Para isso, são utilizadas técnicas de Aprendizagem de Máquina (\textit{Machine Learning}) ou de Processamento de Linguagem Natural (\textit{Natural Language Processing})~\cite{Falaki}. 
Assim, o \textit{chatbot} é treinado com base nas interações dos usuários e consegue aprender com elas, se tornando mais inteligente e preciso ao decorrer deste processo.

As Assistentes Virtuais Inteligentes (\abrv[Assistentes Virtuais Inteligentes]{AVIs}) são um dos exemplos de \textit{chatbots} desse tipo. 
As AVIs são uma das principais tendências de soluções para otimizar o relacionamento entre empresas e consumidores~\cite{DDS}. 
Por meio de mecanismos de Inteligência Artificial, elas aprendem a partir das interações com o consumidor e, com isso, melhoram o seu repertório. 
Assim, são capazes de entender as necessidades do cliente e auxiliá-los da devida maneira na resolução de seus problemas.

\section{Inteligência artificial para \textit{chatbots}}\label{cap:02:sec:02:ia}

\subsection{Aprendizado de máquina}\label{cap:02:sec:02:sub:machine-learning}

De maneira simplificada, Aprendizagem de máquina é a prática de usar algoritmos para coletar dados, aprender com eles, e então fazer uma determinação ou predição sobre alguma atividade específica~\cite{SimpleML}. Assim ao invés de implementar as rotinas de software propriamente ditas, com um conjunto específico de instruções para completar uma tarefa em particular, a máquina é “treinada” usando uma quantidade grande de dados e algoritmos que dão e ela a habilidade de aprender como executar a tarefa. De maneira mais técnica, é um método de análise de dados que automatiza a construção de modelos analíticos~\cite{SASML}. Se baseia na ideia de que sistemas podem aprender com dados, identificar padrões e tomar decisões com o mínimo de intervenção humana.

Existem vários serviços que fornecem e facilitam o uso de tecnologias de aprendizagem de máquina, um deles, por exemplo, é a Amazon Machine Learning.
O Amazon Machine Learning oferece ferramentas e assistentes de visualização que orientam o desenvolvedor durante o processo de criação de modelos de aprendizado de máquina, sem necessidade de aprender tecnologias e algoritmos complexos para o desenvolvimento de tal. 


\subsection{Processamento de Linguagem Natural}\label{cap:02:sec:02:sub:pln}

O Processamento de Linguagem Natural (\abrv[Processamento de Linguagem Natural]{PLN}) é a subárea da IA que estuda a capacidade e as limitações de uma máquina em entender a linguagem dos seres humanos~\cite{PLN1}.
Alguns dos objetivo do PLN é fornecer aos computadores a capacidade de  reconhecer o contexto, fazer análise sintática, semântica, léxica e morfológica, criar resumos, extrair informação, interpretar os sentidos, analisar sentimentos e até aprender conceitos com os textos processados~\cite{PLN1}.

Um dos métodos utilizados do PLN para o desenvolvimento de \textit{chatbots} é transformar uma sentença textual (dado) em informação (intenção e entidades)~\cite{Anatomy}. Em outras palavras, as intenções expressam funcionalidades, entidades expressam parâmetros para a execução de uma funcionalidade. Essas entidades precisam ser cadastradas, de forma a servir à base de conhecimento do PLN. Depois criamos as intenções, onde determinamos frases e sentenças que usarão essas entidades para expressar essas intenções. Assim, de modo simplório, o PLN passa a conseguir interpretar textos completos, textos simples e até mesmo incompletos.

Atualmente, existem vários serviços que dão suporte para a criação de PLN, um desses exemplos é o Wit.ia. Wit.ai é uma plataforma de desenvolvimento de PLN gratuita que transforma a linguagem natural (fala ou escrita) em dados estruturados. Um dos principais motivos para o uso do Wit.ai é por sua simplicidade no processo de criação de aplicativos e dispositivos com os quais as pessoas podem conversar, nesse contexto, na criação de \textit{chatbots}. Com isso, se abstrai a necessidade de aprender todo o processo de desenvolvimento de algoritmos de PLN.


\section{\textit{Chatbots} em plataformas de mensagens instantâneas}\label{cap:02:sec:03:sub:chatbotsmessenger}

Um mensageiro instantâneo consiste em um software que permite que diferentes usuários troquem mensagens, geralmente por escrito, em tempo real. 
Esses mensageiros instantâneos podem possuir alguns recursos adicionais como o envio de arquivos, conversas de áudio, conversas coletivas e até video conferências.
O termo mensageiro instantâneo, no entanto, encontra-se em desuso, sendo agora chamado com mais frequência por plataformas de mensagens instantâneas, ou simplesmente por \textit{messengers}. Podemos citar alguns \textit{messengers} como exemplo, tais como o Whatsapp, Telegram, WeChat, Slack, Facebook Messenger, entre outros.

No que se diz respeito à criação de \textit{chatbots}, alguns desses \textit{messengers} oferecem ferramentas e suporte, sob licenças especificas, para o desenvolvimento em sua plataforma, por meio de \textit{Application Programming Interfaces} (APIs).
Dentre os \textit{messengers}, um dos mais famosos por possuírem tais funcionalidades, é o Telegram.
O Telegram foi lançado em 2015 e possuí cerca de 100 milhões de usuários ativos mensais~\cite{IMaster}, 
É válido ressaltar também, os seus recursos avançados em questão de segurança e criptografia.
O Telegram oferece suporte para desenvolvimento de \textit{chatbots} desde 2015, por meio de sua API baseada no protocolo \abrv[Hypertext Transfer Protocol]{HTTP} (Hypertext Transfer Protocol).

\section{Desenvolvimento de \textit{chatbots}}

No desenvolvimento de \textit{chatbots}, pode-se dizer que existem duas maneiras para a sua criação: utilizando ferramentas de plataformas de \textit{chatbots} ou não. Para melhor explicar a vantagens e desvantagens de cada modo, nas subseções abaixo, estes serão detalhados.

\subsection{Utilizando plataformas de \textit{chatbots}}

A plataformas de \textit{chatbots} são sistemas que oferecem serviços para facilitar a criação de \textit{chatbots} e sua integração com os  \textit{messengers}.
Basicamente, essas plataformas abstraem a necessidade de programar rotinas de códigos complexas, já oferecem recursos ou serviços de IA embutidos e também a tradução da aplicação, aonde o \textit{chatbot} poderá ser exportado e integrado para os mais diversos \textit{messengers}.
O intuito dessas plataformas, é elevar a produtividade dos desenvolvedores responsáveis pela criação do \textit{chatbot}.
Assim, ao facilitar o processo de criação, abstraindo várias etapas complexas e trabalhosas, eles serão capazes de desenvolverem o \textit{chatbot} em menos tempo.
Como exemplo dessas plataformas de \textit{chatbots}, podemos citar a Microsoft Bot Plataform, ChatScript, Pandorabots, Facebook Bots for Messenger, BLIP, entre outros.

Porém, para que possam se utilizar dos recursos mais avançados dessas plataformas, as empresas ou desenvolvedores contratantes devem efetuar os pagamentos, que podem variar de acordo com o tipo de serviço solicitado, podendo estes serviços serem mais caros de acordo com o número de usuários simultâneos que o \textit{chatbot} poderá atender, até, em questões de utilização de recursos mais sofisticados e personalizados para agilizar no desenvolvimento dos \textit{chatbots}. 


\subsection{Não utilizando plataformas de \textit{chatbots}}

Caso o desenvolvedor opte pela não utilização dessas plataformas de \textit{chatbots}, ele poderá customizar a sua aplicação utilizando as tecnologias, ferramentas e serviços que lhe bem convir.
Por exemplo, ele poderá escolher desde de uma linguagem de programação específica até qual serviço de IA (caso seja um \textit{chatbot} de domínio amplo) que ele irá utilizar no projeto.
Como dito na seção \ref{cap:02:sec:03:sub:chatbotsmessenger}, muitos \textit{messengers} oferecem suporte e serviços, por meio de suas APIs, que facilitam o processo de desenvolvimento e integração do \textit{chatbot} em sua plataforma. Assim, o desenvolvedor poderá avaliar dentre os serviços oferecidos por esses \textit{messengers} e escolher qual lhe oferece mais benefícios.

É válido ressaltar a existência de bibliotecas que facilitam ainda mais o de desenvolvimento de \textit{chatbots}. Essas, por sua vez, abstraem algumas etapas no processo de criação e integração dos \textit{chatbots} com os \textit{messengers}. Podemos resumi-las em uma implementação real das regras de uma API de um \textit{messenger}, no contexto de \textit{chatbots}, elas oferecem uma interface mais simplificada para o seu desenvolvimento, utilizando alguma linguagem de programação específica, aonde se abstrai, por exemplo, a necessidade de programar rotinas que possibilitem a comunicação do \textit{chatbot}, via algum protocolo pré-determinado, com a API do \textit{messenger}.

Porém, ao escolher esse modelo criação, o desenvolvedor irá demorar mais tempo durante o processo de desenvolvimento do \textit{chatbot} em comparação aos serviços oferecidos pelas plataformas de \textit{chatbots}. 

\COMMENT{PRECISA DE UM FECHAMENTO MELHOR}