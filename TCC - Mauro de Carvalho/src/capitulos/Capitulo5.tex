% Considera��es finais
\chapter{Considerações finais}

Com o avanço tecnológico, soluções inovadoras e de baixo custo estão surgindo a todo momento. No cenário do atendimento on-line, o uso de \textit{chatbots} é uma opção de baixo custo e alto desempenho. Como tal, o \textit{chatbot} é uma solução possível para atendimento massificado via algum método de conversação, como por exemplo em aplicações de mensagens instantâneas.

No contexto da EV.G, foi desenvolvida a EVA (EV.G Virtual Assistant) é um \textit{chatbot} de domínio amplo, que será capaz de interagir com os alunos vinculados ao EV.G na medida de suas necessidades. EVA irá atender tais solicitações via mecanismo de conversação textual na plataforma de mensagens instantâneas chamada Telegram, no atendimento administrativo de uma secretaria acadêmica. Em teoria, EVA não tem limite de atendimento simultâneo de alunos, nem tão pouco de horário de atendimento, por ser um programa de computador. O que torna ideal para atendimento em ambiente virtual de aprendizagem de alta disponibilidade, que é o caso da EV.G.

Por se tratar de um \textit{chatbot} de domínio amplo, EVA poderá por meio de mecanismos de Inteligência Artificial, que para o seu caso é a PNL, aprender a partir das interações com os alunos e, com isso, melhorar o seu repertório.

\section{Limitações}

O entendimento de EVA ainda é limitado devido ao seu pouco treinamento com base nas interações com os usuários do sistema, e também, devido as poucas intenções cadastradas e compreendidas durante a análise léxica, utilizando o PNL do Wit.ai. É válido ressaltar, que o Wit.ai oferece serviços para treinamento dos PNL desenvolvidos em sua plataforma, porém, devido ao escopo do presente trabalho, o treinamento realizado foi limitado às funcionalidades prioritárias requeridas pelo EV.G.

\section{Trabalhos futuros}

Espera-se que as funcionalidades de EVA possam continuar a serem desenvolvidas e melhoradas com o tempo, para que se possa atender as demandas dos alunos do EV.G, com ainda mais eficiência.

Um dos intuitos da criação da API de EVA, se deu justamente para facilitar que o sistema possa continuar crescendo, seja atendendo em outras plataformas de mensagens instantâneas, estabelecendo de uma interoperabilidade com outros serviços, e até mesmo para facilitar na implementação de novas funcionalidades de EVA. 