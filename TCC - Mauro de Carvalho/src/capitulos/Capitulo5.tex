% Considera��es finais
\chapter{Considerações finais}

Oferecer um serviço massivo sem a devida capacidade de atendimento aos usuários representa um risco para a qualidade e a sustentabilidade do serviço prestado. De um lado, tem-se a insatisfação dos usuários causada pela falta de atendimento, e, de outro, a incapacidade do provedor de colher informações que são estratégicas para a melhoria contínua dos serviços e para a tomada de decisão.

No âmbito da EV.G, observa-se que grande parte das solicitações dos alunos referem-se a assuntos simples, repetitivos e facilmente resolvidos a partir de uma consulta a informações disponíveis em banco de dados. É o caso de dúvidas relativas à emissão de certificados, procedimentos de inscrição, credenciais de acesso, entre outros. Dúvidas qualitativas, referentes à conteúdos, situações inéditas, por exemplo, são raras e podem ser direcionadas para um atendimento de segundo nível.  

Com o avanço tecnológico, soluções inovadoras e de baixo custo estão surgindo a todo momento. No cenário do atendimento on-line, o uso de \textit{chatbots} é uma opção de baixo custo e alto desempenho. Como tal, o \textit{chatbot} é uma solução possível para atendimento massificado via algum método de conversação, como por exemplo em aplicações de mensagens instantâneas.

No contexto da EV.G, foi desenvolvido um \textit{chatbot} de conversação textual compatível com a plataforma mensagens instantâneas chamada Telegram, denominado EVA (EV.G Virtual Assistant). O objetivo geral de EVA é automatizar a interação via texto no atendimento administrativo de primeiro nível a alunos no âmbito da secretaria acadêmica da EV.G.

EVA é um \textit{chatbot} de domínio amplo, que possui a capacidade de interagir com os alunos vinculados a EV.G na medida de suas necessidades. Em teoria, EVA não tem limite de atendimento simultâneo de alunos, nem tão pouco de horário de atendimento, por ser um programa de computador. O que torna ideal para atendimento em ambiente virtual de aprendizagem de alta disponibilidade, que é o caso da EV.G.

Por se tratar de um \textit{chatbot} de domínio amplo, EVA pode por meio de mecanismos de Inteligência Artificial, que para o seu caso utilizasse do PNL, aprender a partir das interações com os alunos e com isso, melhorar o seu repertório.

\section{Limitações}

O entendimento de EVA ainda é limitado devido ao seu pouco treinamento com base nas interações com os usuários do sistema, e também, devido as poucas intenções cadastradas e compreendidas durante a análise léxica, utilizando o PNL do Wit.ai. É válido ressaltar, que o Wit.ai oferece serviços para treinamento dos PNL desenvolvidos em sua plataforma, porém, devido ao escopo do presente trabalho, o treinamento realizado foi limitado às funcionalidades prioritárias requeridas pela EV.G.

\section{Trabalhos futuros}

Espera-se que as funcionalidades de EVA possam continuar a serem desenvolvidas e melhoradas com o tempo, para que se possa atender as demandas dos alunos da EV.G, com ainda mais eficiência.

Um dos intuitos da criação da API de EVA, se deu justamente para facilitar que o sistema possa continuar crescendo, seja atendendo em outras plataformas de mensagens instantâneas, estabelecendo de uma interoperabilidade com outros serviços, e até mesmo para facilitar na implementação de novas funcionalidades de EVA. 