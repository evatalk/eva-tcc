% MIT License
%
% Copyright (c) 2018 José Nascimento <joseaugustodearaujonascimento@gmail.com>
%
% Permission is hereby granted, free of charge, to any person obtaining a copy
% of this software and associated documentation files (the "Software"), to deal
% in the Software without restriction, including without limitation the rights
% to use, copy, modify, merge, publish, distribute, sublicense, and/or sell
% copies of the Software, and to permit persons to whom the Software is
% furnished to do so, subject to the following conditions:
%
% The above copyright notice and this permission notice shall be included in all
% copies or substantial portions of the Software.
%
% THE SOFTWARE IS PROVIDED "AS IS", WITHOUT WARRANTY OF ANY KIND, EXPRESS OR
% IMPLIED, INCLUDING BUT NOT LIMITED TO THE WARRANTIES OF MERCHANTABILITY,
% FITNESS FOR A PARTICULAR PURPOSE AND NONINFRINGEMENT. IN NO EVENT SHALL THE
% AUTHORS OR COPYRIGHT HOLDERS BE LIABLE FOR ANY CLAIM, DAMAGES OR OTHER
% LIABILITY, WHETHER IN AN ACTION OF CONTRACT, TORT OR OTHERWISE, ARISING FROM,
% OUT OF OR IN CONNECTION WITH THE SOFTWARE OR THE USE OR OTHER DEALINGS IN THE
% SOFTWARE.

% resumo em inglês
\begin{resumo}[Abstract]
 \begin{otherlanguage*}{english}
The Escola Virtual de Governo (EV.G) is an initiative that consists of a single portal to offer distance training aimed at public servants and citizens from all over the country.
Launched in 2018, EV.G offers distance courses from different institutions and on different topics related to public administration and citizenship, making it possible to contribute to the training and development of thousands of public servants as well as citizens.
Although it is an environment of mass offer of courses (MOOC), in which user service can be considered a competitive differential, EV.G does not yet have a service center for the students as such.
In the online service scenario, the use of chatbots is a low-cost, high-performance option.
In this context, it was developed by the author together with the EV.G, a textual conversation chatbot compatible with the instant messaging platform called Telegram, called EVA (EV.G Virtual Assistant). The general objective of EVA is to automate the interaction through text in the first level administrative attendance to students within the scope of an academic secretariat.

  %\vspace{\onelineskip}

   \noindent
   {Keywords}: Chatbot. Customer Service. Online Customer Service. MOOC.
 \end{otherlanguage*}
\end{resumo}
%--