% MIT License
%
% Copyright (c) 2018 José Nascimento <joseaugustodearaujonascimento@gmail.com>
%
% Permission is hereby granted, free of charge, to any person obtaining a copy
% of this software and associated documentation files (the "Software"), to deal
% in the Software without restriction, including without limitation the rights
% to use, copy, modify, merge, publish, distribute, sublicense, and/or sell
% copies of the Software, and to permit persons to whom the Software is
% furnished to do so, subject to the following conditions:
%
% The above copyright notice and this permission notice shall be included in all
% copies or substantial portions of the Software.
%
% THE SOFTWARE IS PROVIDED "AS IS", WITHOUT WARRANTY OF ANY KIND, EXPRESS OR
% IMPLIED, INCLUDING BUT NOT LIMITED TO THE WARRANTIES OF MERCHANTABILITY,
% FITNESS FOR A PARTICULAR PURPOSE AND NONINFRINGEMENT. IN NO EVENT SHALL THE
% AUTHORS OR COPYRIGHT HOLDERS BE LIABLE FOR ANY CLAIM, DAMAGES OR OTHER
% LIABILITY, WHETHER IN AN ACTION OF CONTRACT, TORT OR OTHERWISE, ARISING FROM,
% OUT OF OR IN CONNECTION WITH THE SOFTWARE OR THE USE OR OTHER DEALINGS IN THE
% SOFTWARE.

\setlength{\absparsep}{18pt} % ajusta o espaçamento dos parágrafos do resumo
\begin{resumo}
A Escola Virtual de Governo (EV.G) é uma iniciativa que consiste em um portal único para oferta de capacitação a distância voltado a servidores públicos e cidadãos de todo o país.
Lançada em 2018, a EV.G oferece cursos a distância de diferentes instituições e em diferentes temáticas ligadas à administração pública e à cidadania, viabilizando o desafio de contribuir para a formação e o desenvolvimento de milhares de servidores públicos bem como de cidadãos.
Apesar de constituir-se como um ambiente de oferta massiva de cursos (MOOC), em que o atendimento ao usuário pode ser considerado um diferencial competitivo, a EV.G ainda não dispõe de uma central de atendimento aos alunos propriamente dita.
No cenário do atendimento on-line, o uso de \textit{chatbots} é uma opção de baixo custo e alto desempenho. 
Neste contexto, foi desenvolvido pelo autor juntamente com a EV.G, um \textit{chatbot} de conversação textual compatível com a plataforma de mensagens instantâneas chamada Telegram, denominado EVA (EV.G Virtual Assistant). O objetivo geral de EVA é automatizar a interação via texto no atendimento administrativo de primeiro nível a alunos no âmbito de uma secretaria acadêmica.

  %\vspace{\onelineskip}

  \noindent
  {Palavras-chave}: Chatbot. Atendimento ao cliente. Atendimento on-line. MOOC.
\end{resumo}
%--