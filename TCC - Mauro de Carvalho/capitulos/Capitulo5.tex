% MIT License
%
% Copyright (c) 2018 José Nascimento <joseaugustodearaujonascimento@gmail.com>
%
% Permission is hereby granted, free of charge, to any person obtaining a copy
% of this software and associated documentation files (the "Software"), to deal
% in the Software without restriction, including without limitation the rights
% to use, copy, modify, merge, publish, distribute, sublicense, and/or sell
% copies of the Software, and to permit persons to whom the Software is
% furnished to do so, subject to the following conditions:
%
% The above copyright notice and this permission notice shall be included in all
% copies or substantial portions of the Software.
%
% THE SOFTWARE IS PROVIDED "AS IS", WITHOUT WARRANTY OF ANY KIND, EXPRESS OR
% IMPLIED, INCLUDING BUT NOT LIMITED TO THE WARRANTIES OF MERCHANTABILITY,
% FITNESS FOR A PARTICULAR PURPOSE AND NONINFRINGEMENT. IN NO EVENT SHALL THE
% AUTHORS OR COPYRIGHT HOLDERS BE LIABLE FOR ANY CLAIM, DAMAGES OR OTHER
% LIABILITY, WHETHER IN AN ACTION OF CONTRACT, TORT OR OTHERWISE, ARISING FROM,
% OUT OF OR IN CONNECTION WITH THE SOFTWARE OR THE USE OR OTHER DEALINGS IN THE
% SOFTWARE.

% Considera��es finais
\chapter{Considerações finais}

Oferecer um serviço massivo sem a devida capacidade de atendimento aos usuários representa um risco para a qualidade e a sustentabilidade do serviço prestado. De um lado, tem-se a insatisfação dos usuários causada pela falta de atendimento, e, de outro, a incapacidade do provedor de colher informações que são estratégicas para a melhoria contínua dos serviços e para a tomada de decisão.

No âmbito da EV.G, observa-se que grande parte das solicitações dos alunos referem-se a assuntos simples, repetitivos e facilmente resolvidos a partir de uma consulta a informações disponíveis em banco de dados. É o caso de dúvidas relativas à emissão de certificados, procedimentos de inscrição, credenciais de acesso, entre outros. Dúvidas qualitativas, referentes à conteúdos, situações inéditas, por exemplo, são raras e podem ser direcionadas para um atendimento de segundo nível.  

Com o avanço tecnológico, soluções inovadoras e de baixo custo estão surgindo a todo momento. No cenário do atendimento on-line, o uso de \textit{chatbots} é uma opção de baixo custo e alto desempenho. Como tal, o \textit{chatbot} é uma solução possível para atendimento massificado via algum método de conversação, como por exemplo em aplicações de mensagens instantâneas.

No contexto da EV.G, foi desenvolvido um \textit{chatbot} de conversação textual compatível com a plataforma mensagens instantâneas chamada Telegram, denominado EVA (EV.G Virtual Assistant). O objetivo geral de EVA é automatizar a interação via texto no atendimento administrativo de primeiro nível a alunos no âmbito da secretaria acadêmica da EV.G.

EVA é um \textit{chatbot} de domínio amplo, que possui a capacidade de interagir com os alunos vinculados a EV.G na medida de suas necessidades. Em teoria, EVA não tem limite de atendimento simultâneo de alunos, nem tão pouco de horário de atendimento, por ser um programa de computador. O que torna ideal para atendimento em ambiente virtual de aprendizagem de alta disponibilidade, que é o caso da EV.G.

Por se tratar de um \textit{chatbot} de domínio amplo, EVA pode por meio de mecanismos de Inteligência Artificial, que para o seu caso utilizasse do PNL, aprender a partir das interações com os alunos e com isso, melhorar o seu repertório.



\section{Limitações}

O entendimento de EVA ainda é limitado devido ao seu pouco treinamento com base nas interações com os usuários do sistema, e também, devido as poucas intenções cadastradas e compreendidas durante a análise léxica, utilizando o PNL do Wit.ai. É válido ressaltar, que o Wit.ai oferece serviços para treinamento dos PNL desenvolvidos em sua plataforma, porém, devido ao escopo do presente trabalho, o treinamento realizado foi limitado às funcionalidades prioritárias requeridas pela EV.G.



\section{Trabalhos futuros}

Inicialmente, como primeira proposta de trabalhos futuros, seria a implantação do sistema de EVA em algum servidor. Buscando, por exemplo, maneiras eficiêntes para o armazenamento e também para o processamento do sistema.

Um dos intuitos da criação da API de EVA, se deu justamente para facilitar que o sistema possa continuar crescendo, podendo este atender em outras plataformas de mensagens instantâneas, tais como Facebook messenger ou até mesmo na própria plataforma da EV.G, permitindo aos alunos escolherem aquelas que melhor se adequem a suas realidades.

No que se diz respeito à usabilidade de EVA, poderá ser feito algum estudo sobre a experiência do usuário durante a interação com ela, com o intuito de melhorar ainda mais no atendimento aos alunos vinculados a EV.G. 

Poderão ser desenvolvidas novas funcionalidades que facilitem a modificação e adição de intenções compreendidas por EVA, e também, no que diz respeito ao seu repertório de respostas ao usuário. De modo a permitir que pessoas sem o conhecimento de programação de computadores, possam estar ajudando na expansão e nas melhorias de EVA.

Existe também um anseio por parte da EV.G de que as funcionalidades de EVA possam continuar a serem desenvolvidas, de modo que futuramente EVA possa ajudar proativamente na tutoria dos cursos oferecidos na plataforma. 